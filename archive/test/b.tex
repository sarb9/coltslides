\documentclass[tikz,border=10pt]{standalone}
\usepackage{pgfplots}
\pgfplotsset{compat=1.18}

%-------- plot params ----------
\newcommand\epsval{0.2}        % ε ∈ [0,1/2]
\def\thetazero{1}              % θ_x component
\def\thetaone{0}               % θ_y component
\pgfmathsetmacro\radtest{1+2*\epsval}     % threshold 1+2ε
\pgfmathsetmacro\sqrtt{sqrt(\radtest)}    % √(1+2ε)
 
\begin{document}
\begin{tikzpicture}
    \begin{axis}[
            view={45}{90},
            width=11cm,height=8cm,
            axis lines=center,
            xlabel={$x$},ylabel={$y$},
            xtick=\empty,ytick=\empty, ztick=\empty,
            z axis line style={opacity=0},
            domain=0:1, y domain=-180:180,   % r∈[0,1], φ in degrees
            variable=r, variable y=phi,
            samples=21, samples y=51,
            colormap/viridis,
        ]
        \addplot3[
            surf, shader=flat,
            draw=black!60, opacity=0.85,
        ]
        (
        {r*cos(phi)},
        {r*sin(phi)},
        {
                ( ((r*cos(phi))^2 + (r*sin(phi))^2) > 1 ) ? (NaN) :
                ( (((r*cos(phi))-\thetazero)^2 + ((r*sin(phi))-\thetaone)^2) >= \radtest ) ?
                ( \epsval + 0.5*((r*cos(phi))^2 + (r*sin(phi))^2) ) :
                ( \thetazero*(r*cos(phi)) + \thetaone*(r*sin(phi)) - 1 + \sqrtt*sqrt( ((r*cos(phi))-\thetazero)^2 + ((r*sin(phi))-\thetaone)^2 ) )
            }
        );
    \end{axis}
\end{tikzpicture}
\end{document}
